\subsubsection{Instala��o e configura��o do PostgreSQL}

Este banco de dados est� dispon�vel em praticamente todas as distribui��es Linux atuais: Conectiva, Red Hat, Mandrake, Debian, SuSE. Durante todo este projeto, utilizou-se o Conectiva 6.0 (at� o primeiro semestre de 2001) e Conectiva 7.0 (a partir do segundo semestre de 2001).

Primeiramente, � necess�rio estar registrado ("logado") no sistema como o usu�rio \textbf{root}. A seguir, dando in�cio � instala��o do PostgreSQL atrav�s do apt-get, execute o comando \textbf{apt-get install postgresql}. Todos os pacotes necess�rios v�o ser pegos do reposit�rio de pacotes RPM utilizado pelo apt e instalados automaticamente. Agora falta apenas configurar o PostgreSQL para receber cone��es TCP/IP. Isto � essencial tendo em vista que os drivers JDBC utilizam-se deste protocolo para se conectarem aos bancos de dados. Edite o arquivo \textbf{/var/lib/pgsql/data/postgresql.conf} e altere a linha \textbf{tcpip_socket = false} para \textbf{tcpip_socket = on }. Inicie o banco de dados atrav�s do comando \textbf{/etc/init.d/postgresql start}. Se n�o ocorreu nenhum problema, o postgresql deve estar funcionando, aceitando conex�es TCP na porta 5432.

Continuando a configura��o, crie a base de dados (tamb�m chamada no PostgreSQL de \textit{catalog}). Primeiro, criemos um usu�rio espec�fico para a CoolCase:

\tiny
\begin{verbatim}
	su - postgres
	createdb test
	psql test
	create user coolcase createdb with password 'coolcase'
	create database coolcase
	grant all on coolcase to coolcase
\end{verbatim}



