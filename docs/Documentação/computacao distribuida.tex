\documentclass[a4paper,11pt]{article}

\usepackage{report}

\author{Marco Aur�lio Graciotto Silva}
\title{Computa��o distribu�da}

\begin{document}

\maketitle

\tableofcontents

\section{Introdu��o}

Com o advento do acesso a Internet � alta velocidade, computa��o distribu�da passou de um del�rio da inform�tica para um requisito do mundo real. Ela permite que o custo de manuten��o do software caia, a utiliza��o de maneira mais eficiente dos diversos computadores conectados � rede, redund�ncia, escalabilidade. Por�m, h� um grande problema: que tecnologia utilizar para desenvolver solu��es distribu�das?

Na �ltima d�cada foram desenvolvidas v�rias solu��es. As primeiras foram ToolTalk e o RPC, da SUN. Seus maiores problemas s�o (...), conforme [NEV99]. Outras solu��es mais recentes s�o DCOM e CORBA.





\end{document}