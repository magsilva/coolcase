Dep�sitos s�o aplica��es que possibilitam o armazenamento de informa��es. Erroneamente, poderiamos dizer que eles s�o nada mais que sistemas de bancos de dados. Mas, na verdade, eles oferecem muito mais. Um dep�sito permite um acesso mais transparente as informa��es nele armazenadas, pode possuir mecanismos de controle de vers�o, sua interface � de mais alto n�vel. Ele acaba por utilizando um SBD para armazenar as informa��es, mas este n�o pode ser utilizado diretamente pelas aplica��es, o acesso aos dados faz-se exclusivamente pelo dep�sito.

Foram estudados alguns dep�sitos: o desenvolvido pelo projeto NIST Design Repository, o Unisys UREP e o Rational ClearCase. Todos eles possuem ferramentas que tornam o processo de engenharia de software muito mais conciso e eficiente, armazenando os dados e fazendo seu controle de vers�es automaticamente, gerenciando os acessos executando restri��es conforme o caso. Tais como SBDs, possuem m�todos para definir os tipos de dados na base. O mecanismo � independente do SQL e suas restri��es para a manipula��o de objetos, empregando documentos MOF para gerenciar e definir os metadados. Esta linguagem foi criada pela OMG e � amplamente utilizada na defini��o de metadados, sendo um padr�o de mercado. A ado��o deste padr�o permite grandes possibilidades, como o emprego de XMI para transportar os dados entre dep�sitos e suas aplica��es. Isto significa interc�mbio de dados de maneira simples e funcional entre aplica��es distintas, algo at� ent�o impens�vel na tarefa de engenharia de software.
